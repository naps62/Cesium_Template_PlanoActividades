\section{Introdução}

\subsection{Contextualização}
\indent Este documento apresenta o plano de actividades para o mandato de 2010/2011 do Centro de Estudantes de Engenharia Informática da Universidade do Minho (CeSIUM). Por forma a fazer uma introdução ao documento é apresentado uma contextualização do CeSIUM, dos seus órgãos para este mandato, assim como a motivação dos mesmos e os objectivos que esperam cumprir. As actividades aqui apresentadas estão susceptíveis a alteração de datas e de variações de orçamento.

\subsection{CeSIUM 2010/2011}
\indent A nova direcção do CeSIUM tomou posse no dia 19 de Outubro de 2010, onde Roberto Machado foi reeleito como presidente do CeSIUM para mais um ano lectivo. O CeSIUM para 2010/2011 pretende continuar o espírito de evolução para o núcleo que foi implementado pelas listas anteriores e permanecer na vanguarda do que melhor é feito por núcleos de estudantes na Universidade do Minho. Queremos continuar projectos de sucesso no passado e implementar novos horizontes para este centro de estudantes, investindo em projectos de visibilidade a nível nacional.

\subsection{Motivação}
\indent Conscientes da dimensão da engenharia informática na Universidade do Minho, da dimensão do núcleo, e do peso que ao longo dos anos a LESI carregou como fonte de criação de massa cinzenta para a região norte de Portugal e para o desenvolvimento de centenas de empresas ao longo do país, temos uma motivação que é elevar o nome do nosso curso e mestrados agregados. Queremos espalhar o nome dos nosso curso pelo território nacional, chegar aos jovens que estão a tentar ingressar no ensino superior, passar-lhes a mensagem que boa informática é no Minho, é em Braga e começasse em LEI. 

\indent Aliado a esta vontade de uma imagem externa da engenharia informática do Minho, queremos também trabalhar numa boa imagem interna fornecendo um conjunto de actividades e serviços aliciantes, principalmente para os nossos alunos, mas também, sempre que se justifique, para toda a comunidade académica. Por forma a finalizar a motivação desta equipa, salientar o principal foco deste centro de estudantes, criar actividades e espaços que aumentam o já excelente espirito de grupo, solidariedade, amizade, etc entre os alunos do curso.

\subsection{Objectivos do Mandato}
Os objectivos a que esta direcção tomou como essenciais são os seguintes:
\begin{itemize}
\item Aumentar a interacção entre os alunos dos diferentes ano e áreas;
\item Diminuir a distância professor-aluno, realizando actividades que obriguem a uma interacção informal entre os mesmos;
\item Pretendemos melhorar a imagem junto do departamento, trazendo e apoiando actividades do Departamento de Informática;
\item Continuar a melhorar as condições da sala, para cada vez mais os sócios do CeSIUM, sintam que têm uma casa dentro de portas;
\item Estruturar o CeSIUM de normas de actividade, regulamentos e outros documentos que contribuam para uma maior facilidade de trabalhos para as próximas direcções;
\item Remodelar a Imagem do CeSIUM, novo logotipo e novo site;
\item Realizar e planear eventos a nível regional e nacional;
\item Levar os nossos alunos a participar em evento de renome, disponibilizando transportes e tratando da logistica;
\item Criar parcerias com empresas da zona de Braga, em que se possa trazer benefícios para os sócios do CeSIUM;
\item Aumentar a formação dos nossos colegas;
\item Realizar um conjunto de eventos recreativos que tragam boa disposição ao nosso ambiente académico;
\item Apostar ainda mais na Semana da LEI;
\item Reforçar a nossa capacidade financeira, para que no futuro, projectos mais ambiciosos possam surgir.
\end{itemize}

\subsection{Estrutura do plano de Actividades}
\indent Este plano de actividades, começa por fazer uma introdução do que o CeSIUM pretende-se no presente ano lectivo. Posteriormente é apresentado todas as actividades da direcção do CeSIUM, seguido de todos os seus departamentos. No final do documento é apresentado os custos de funcionamento de todo o CeSIUM e um conjunto de bens materiais que pretendemos adquirir no corrente ano.

\newpage
