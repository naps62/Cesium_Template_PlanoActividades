\subsection{Actividade: Workshop  de Introdução ao Prolog} %alterar nome

\subsubsection*{Objectivos da actividade:}
Ainda no tempo da LESI, existia uma cadeira chamada Paradigmas de Programação 3 (PP3), onde se introduzia o paradigma a linguagem de programação Prolog. Os conhecimentos adquiridos nesta disciplina eram de seguida aproveitados em Representação do Conhecimento (RC), uma disciplina onde eram extendidos os conceitos de programação lógica. Com a reformulação do curso, RC (mais ou menos remodelada) passou a denominar-se Sistemas de Representação do Conhecimento e Raciocínio (SRCR), mas PP3 deixou de ser leccionada, obrigando os docentes de SRCR a gastarem algumas aulas a ensinar Prolog.

Com o objectivo de auxiliar os docentes e alunos de SRCR, o CeSIUM está a planear para breve um Workshop de Introdução ao Prolog.
\subsubsection*{Responsáveis:}
\begin{itemizedash}
	\item{André Santos}
	\item{Rafael Silva}
\end{itemizedash}

\paragraph{Publico Alvo: }
Alunos do 3º ano da LEI

\paragraph{Esta actividade já se realizou em anos anteriores?}
Não

\paragraph{Número de Participantes Esperado:}
40

\paragraph{Data:} Março 2011

\subsubsection*{Parceiros:}
\begin{itemizedash}
    \item{Departamento de Informática}
	\item{/log/cesium}
    \item{António P. O. C. Santos (orador)}
\end{itemizedash}

\subsubsection*{Orçamento:}

\begin{center}

\resizebox{0.9\textwidth}{!}{

\begin{tabular}{| c | l | r | l | r |}

\hline

\multirow{2}{*}{Actividade} & \multicolumn{2}{ | c | }{Despesas} & \multicolumn{2}{ | c | }{Receitas} \\ \cline{2-5}
                            & Designação                            & Valor           & Designação                          & Valor             \\ \hline

\multirow{3}{*}{Workshop Prolog}
							& Aluguer de Sala	 	& 100 \euro		& Departamento de Informática 	& 100 \euro			\\ \cline{2-5}
							&						&				&								&					\\ \cline{2-5}
							& \textbf{Total}		& 100 \euro		& \textbf{Total}				& 100 \euro			\\ \cline{1-5} \hline
\end{tabular}
}
\end{center}


\vspace{20pt}
