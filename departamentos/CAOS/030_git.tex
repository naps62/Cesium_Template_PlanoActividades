\subsection{Actividade: Tutorial Git} %alterar nome

\subsubsection*{Objectivos da actividade:}
No desenvolvimento de qualquer trabalho colaborativo (ou mesmo individual), o controlo de versões é uma vertente essencial que assegura protecção contra erros e melhor gestão do trabalho em paralelo. Git é um sistema distribuído de controlo de versões, open source e concebido para suportar desde projectos pequenos até projectos de larga escala mantendo a rapidez e a eficiência.

Apesar do carácter fundamental deste tipo de ferramentas, habitualmente os alunos da LEI não as utilizam nos primeiros anos, por desconhecimento do que são, para o que servem e quais as suas vantagens. Desta forma, será dada uma sessão sobre Git especialmente direccionada aos alunos dos primeiros anos da LEI.

\subsubsection*{Responsáveis:}
\begin{itemizedash}
	\item{Miguel Regedor}
	\item{Miguel Palhas}
\end{itemizedash}

\paragraph{Publico Alvo: }
Alunos dos primeiros anos da LEI, mas aberta a todos os alunos interessados.

\paragraph{Esta actividade já se realizou em anos anteriores?}
Sim

\paragraph{Número de Participantes Esperado:}
30

\paragraph{Data:} Março 2011

\subsubsection*{Parceiros:}
\begin{itemizedash}
    \item{Departamento de Informática}
	\item{/log/cesium}
	\item{Miguel Regedor (orador)}
\end{itemizedash}

\subsubsection*{Orçamento:}

\begin{center}

\resizebox{0.9\textwidth}{!}{

\begin{tabular}{| c | l | r | l | r |}

\hline

\multirow{2}{*}{Actividade} & \multicolumn{2}{ | c | }{Despesas} & \multicolumn{2}{ | c | }{Receitas} \\ \cline{2-5}
                            & Designação                            & Valor           & Designação                          & Valor             \\ \hline

\multirow{2}{*}{Tutorial Git}  & Aluguer de Salas                            & 100.00 \euro     &   Departamento de Informática    &  100.00                      \euro                 \\ \cline{2-5}
                            & \multicolumn{1}{c|}{\textbf{Total}}   & 100.00 \euro    & \multicolumn{1}{c|}{\textbf{Total}} & 100.00 \euro        \\ \hline
\end{tabular}
}
\end{center}


\vspace{20pt}
