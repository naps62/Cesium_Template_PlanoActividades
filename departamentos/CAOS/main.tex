\section{Departamento: CAOS}

\subsection{Responsáveis de Departamento}
\begin{itemizedash}
	\item André Santos
	\item Miguel Palhas
\end{itemizedash}


\indent Inicialmente criado como o Centro de Apoio ao Open Source, a missão principal do CAOS é apoiar, promover e incentivar a utilização de software livre. Nos últimos anos tem sido ainda responsável pela gestão das infraestruturas informáticas do CeSIUM.

Com o objectivo de aumentar o interesse da comunidade pelas alternativas \textit{open source} existentes, todos os anos são organizadas várias palestras/workshops. Algumas destas, pelo seu carácter mais introdutório e por coincidirem com conteúdos de unidades curriculares de LEI, são especialmente dirigidas aos alunos da licenciatura, cumprindo um papel fulcral de apoio e suporte aos sócios mais jovens. Outras sessões possuem um caráter mais avançado, focando-se em temas mais complexos, tendo como objectivo proporcionar aos participantes conhecimentos avançados de áreas e tecnologias que lhes possam vir a ser úteis no futuro profissional.

A gestão do parque informático consiste na manutenção dos computadores e servidores que o CeSIUM possui, e no planeamento e implementação das soluções que melhor permitam colocar estas infraestruturas ao serviço dos alunos. Nesta categoria incluem-se actividades tais como a manutenção de um \textit{mirror} de software livre, a disponibilização de serviços informáticos para os sócios, a gestão de portais web, entre outros. 

De seguida apresentamos detalhada e cronologicamente as actividades que o CAOS planeia promover no decorrer deste ano lectivo.
