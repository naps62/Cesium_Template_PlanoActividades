%inicio semana da lei
\subsection{Semana da LEI 2011}
%
\subsubsection**{Objectivos da actividade:} %objectivos???
A Semana da LEI é a actividade que envolve maior força organizativa por parte de todos os colaboradores do Cesium.
\paragraph{}É uma semana de com vastos eventos culturais, desportivos, recreativos e pedadógicos, em que se pretende celebrar a grandiosidade do nosso grande curso.
%lista de actividades
\subsubsection**{Lista de Actividades:}
De seguida vamos apresentar as actividades que irão ocorrer durante a edição de 2011 da Semana da LEI.
Ano após anos, temos procurado um programa capaz de corresponder às amplas exigências dos nossos alunos. Este ano, em seguimento do conseguido nas edições anteriores, apresentamos um programa bastante vasto.
%segunda feira
\subsubsection*{Segunda-feira:}
Na segunda-feira a nossa semana irá ser iniciada com a actuação de um grupo cultural da Universidade do Minho. De seguida irão decorrer actividades lúdicas relacionadas com a área da informática. Durante a noite, no Bar Académico, irão ser desenvolvidos alguns torneios lúdicos.
%
\paragraph{Actuação dos iPum:}
O som dos tambores deste grupo grupo cultural da Universidade do Minho, proclamará o início da nossa grande semana. Irão realizar uma actuação na cantina da Universidade do Minho, e de seguida, irão actuar pelo \emph{campus}.
\begin{description}
\item[Data:] 14.03.2011
\item[Hora:] 14h00
\end{description}
%
\paragraph{Torneio de Robocode}
Esta actividade é a actividade lúdica mais \emph{nerd} que teremos nesta semana.
Robocode é um jogo educativo \emph{open source} criado por Mathew Nelson (originalmente disponibilizado pela IBM).
O jogo foi concebido para ajudar as pessoas a aprender a programar em Java e desfrutar da experiência. É muito fácil começar: um robô simples pode ser escrito em apenas alguns minutos, mas aperfeiçoar um pode levar meses ou mais.
Motivados pela procura que esta actividade teve na edição anterior, decidímos trazer de novo esta actividade.
\begin{description}
\item[Data:] 14.03.2011
\item[Hora:] 16h00
\end{description}
\begin{description}
\item[Despesas:] Iremos apostar em bons e atractivos prémios na área das TI, de modo a conseguírmos uma boa participação nesta actividade.
\begin{itemize}
\item Prémios
\begin{itemize}
\item 1$\,^{\circ}$ lugar:
\item 2$\,^{\circ}$ lugar:
\item 3$\,^{\circ}$ lugar:
\end{itemize}
\item Apoios
\begin{itemize}
\item iLook
\item Fnac
\item Megatronica
\end{itemize}
\end{itemize}
\end{description}
%
\paragraph{Torneio de Matraquilhos}
Iremos após o jantar, realizar um Torneio de Matraquilhos no Bar Académico. Este torneio irá ter prémios para os três primeiros lugares. Todos os alunos da Universidade do Minho poderão participar neste evento.
\begin{description}
\item[Data:] 14.03.2011
\item[Hora:] 22h00
\item[Local:] Bar Académico (Braga)
\item[Despesas:] Como pretendemos ter um bom número de participantes, vamos apostar em bons e atractivos prémios.
\begin{itemize}
\item Prémios
\begin{itemize}
\item 1$\,^{\circ}$ Lugar:
\item 2$\,^{\circ}$ Lugar:
\item 3$\,^{\circ}$ Lugar:
\end{itemize}
\item Apoios
\begin{itemize}
\item Bilhares Oásis
\item BA
\item \dots  
\end{itemize}
\end{itemize}
\end{description}
%
\paragraph{Torneio de Ping Pong:}
Paralelamente ao Torneio de Matraquilhos iremos realizar um Torneio de Ping Pong. Irá ter prémios para os três primeiros lugares. Todos os alunos da Universidade do Minho, à semelhança do Torneio de Matraquilhos, poderão participar neste evento.
\begin{description}
\item[Data:] 14.03.2011
\item[Hora:] 22h00
\item[Local:] Bar Académico (Braga)
\item[Despesas:] Como pretendemos ter um bom número de participantes, vamos apostar em bons e atractivos prémios.
\begin{itemize}
\item Prémios
\begin{itemize}
\item 1$\,^{\circ}$ Lugar:
\item 2$\,^{\circ}$ Lugar:
\item 3$\,^{\circ}$ Lugar:
\end{itemize}
\item Apoios
\begin{itemize}
\item Bilhares Oásis
\item BA
\item \dots  
\end{itemize}
\end{itemize}
\end{description}
%
\paragraph{Karaoke no BA:}
Recordando, o passado, iremos organizar um torneio de Karaoke com grandes êxitos do século passado. Sem dúvida que será um actividade em que todos os participantes se vão divertir imenso. Existirá bebida para todos os participantes oferecida pela organização.
\begin{description}
\item[Data:] 14.03.2011
\item[Hora:] 23h30
\item[Local:] Bar Académico (Braga)
\item[Despesas:] Com este evento pretendemos oferecer a todos os participantes e pessoas que se encontrem no Bar Académico um bom final de noite. Para tal vamos oferecer bons e atractivos prémios e ainda bebida para todos os participantes na actividade.
\begin{itemize}
\item Prémios
\begin{itemize}
\item 1$\,^{\circ}$ Lugar:
\item 2$\,^{\circ}$ Lugar:
\item 3$\,^{\circ}$ Lugar:
\end{itemize}
\item Bebidas: barril de cerveja
\item Apoios
\begin{itemize}
\item Fnac
\item BA
\item \dots  
\end{itemize}
\end{itemize}
\end{description}
%terca feira
\subsubsection*{Terça-feira}
No segundo dia de actividades, teremos durante a tarde a realização de vários torneios de jogos electrónicos muito queridos dos informáticos. Durante a noite teremos divertimento garantido com a realização de um rally das tascas, seguido de festa no Bar Académico.
%
\paragraph{Torneio de PES}
Torneio de um dos jogos mais populares da actualidade. O PES deve ser o jogo electrónico mais popular e dos mais jogados, pelo que esperamos ter um bom número de participantes.
\begin{description}
\item[Data:] 15.03.2011
\item[Hora:] 16h00
\item[Local:] Departamento de Informática
\item[Despesas:] prémios para os 3 primeiros lugares.
\begin{itemize}
\item Prémios
\begin{itemize}
\item 1$\,^{\circ}$ Lugar:
\item 2$\,^{\circ}$ Lugar:
\item 3$\,^{\circ}$ Lugar:
\end{itemize}
\item Apoios
\begin{itemize}
\item Game
\item Fnac
\item \dots  
\end{itemize}
\end{itemize}
\end{description}
%
\paragraph{Torneio de Guitar Hero}
Guitar Hero é um jogo muito popular em que os dotes musicais dos jogadores são postos à prova.
\begin{description}
\item[Data:] 15.03.2011
\item[Hora:] 16h00
\item[Local:] Departamento de Informática
\item[Despesas:] prémios para os primeiros 3 lugares.
\begin{itemize}
\item Prémios
\begin{itemize}
\item 1$\,^{\circ}$ Lugar:
\item 2$\,^{\circ}$ Lugar:
\item 3$\,^{\circ}$ Lugar:
\end{itemize}
\item Apoios
\begin{itemize}
\item Game
\item Fnac
\item \dots  
\end{itemize}
\end{itemize}
\end{description}
%
\paragraph{Torneio de BUZZ}
Vamos testar a cultura dos nossos alunos com um jogo divertido que coloca o nosso conhecimento geral à prova.
\begin{description}
\item[Data:] 09.03.2010
\item[Hora:] 16h00
\item[Local:] Departamento de Informática
\item[Despesas:] prémios para os 3 primeiros lugares.
\begin{itemize}
\item Prémios
\begin{itemize}
\item 1$\,^{\circ}$ Lugar:
\item 2$\,^{\circ}$ Lugar:
\item 3$\,^{\circ}$ Lugar:
\end{itemize}
\item Apoios
\begin{itemize}
\item Game
\item Fnac
\item \dots  
\end{itemize}
\end{itemize}
\end{description}
%
\paragraph{Rally das Tascas}
Esta actividade será realizada em colaboração com a Comissão de Festas de LEI 10/11. Irá decorrer pela noite dentro terminando no Bar Académico.
\begin{description}
\item[Data:] 15.03.2011
\item[Hora:] 22h00
\item[Local:] Bares perto da Universidade do Minho (Braga)
\item[Despesas:] prémios para os 3 primeiros lugares.
\begin{itemize}
\item Prémios
\begin{itemize}
\item 1$\,^{\circ}$ Lugar:
\item 2$\,^{\circ}$ Lugar:
\item 3$\,^{\circ}$ Lugar:
\end{itemize}
\item Apoios
\begin{itemize}
\item Game %WTF?? FIXME
\item Fnac %WTF?? FIXME
\item \dots  
\end{itemize}
\end{itemize}
\end{description}
%quarta feira
\subsubsection*{Quarta-feira}
Este dia é dedicado em especial à Informática e a todos as pessoas do Departamento de Informática da Universidade do Minho.
Começamos a tarde com uma sessão pedagógica com oradores exteriores ao DI. Mais tarde teremos alguns jogos de futebol divertidos com o failplay como jogador principal. Neste dia haverá um grande Jantar de Curso, onde estão convidados todos os alunos representados pelo CeSIUM, assim como todo o corpo docente do DI e seus funcionários. Para terminar o dia em grande teremos uma fantástica festa no Sardinha Biba.
%
\paragraph{Sessão pedagógica}
Nesta actividade teremos oradores convidados para discursar sobre temas ligados ao mundo informático embora não sejam pessoas particularmente envolvidas no mesmo. Esta sessão irá assentar num tema ainda a definir.
\begin{description}
\item[Data:] 16.03.2011
\item[Hora:] 14h30
\item[Local:] CP1 A4
\item[Despesas:] Lembranças para os oradores e possivelmente um lanche para o \emph{coffee break}.
\begin{itemize}
\item Apoios
\begin{itemize}
\item Sapo
\item Sun
\item \dots  
\end{itemize}
\end{itemize}
\end{description}
%
\paragraph{Jogos de Futebol}
A meio da semana, vamos realizar três jogos de futebol de salão. O primeiro será uma equipa do CeSIUM contra uma equipa do NECC, o outro nucleo do DI. Depois será realizado um jogo de futebol feminino de uma equipa de LEI contra LCC. Para finalizar teremos um jogo CeSIUM contra DI, esta ultima uma equipa de docentes e funcionários do DI.
\begin{description}
\item[Data:] 16.03.2011
\item[Hora:] 16h30
\item[Local:] Pavilhão Polidesportivo da Universidade do Minho (\emph{campus} de Gualtar)
\item[Despesas:] Pavilhão (4h): 40\euro; lembranças para todos os participantes.
\begin{itemize}
\item Apoios
\begin{itemize}
\item Tasquinha do Careca
\item TPJM
\item \dots  
\end{itemize}
\end{itemize}
\end{description}
%
\paragraph{Jantar de Curso}
Na noite de quarta-feira, iremos promover um jantar de convívio aberto a todos os alunos representados pelo CeSIUM, e professores do DI. De modo a conseguirmos uma maior adesão a este evendo, o CeSIUM irá suportar parte do custo do jantar.\begin{description}
\item[Data:] 16.03.2011
\item[Hora:] 20h30
\item[Local:] Restaurante Os Afonsos
\item[Despesas:] Ajuda de custos: 200\euro
\end{description}
%
\paragraph{Festa no Sardinha}
Em colaboração com a Comissão de Festas de LEI, iremos fazer uma festa em nome do CeSIUM com a participação de outros cursos da Universidade do Minho. Essa festa irá decorrer no Sardinha Biba, e a diversão estará garantida!
\begin{description}
\item[Data:] 16.03.2011
\item[Hora:] 23h59
\item[Local:] Sardinha Biba
\end{description}
%quinta feira
\subsubsection*{Quinta-feira}
No quarto dia desta edição da Semana da LEI, teremos um dia mais descontraído, em que as pessoas poderão relaxar assistir, e participar em pequenas palestras que esperamos serem bastante interessantes. À noite poder-se-á assistir a um filme numa agradável sessão ao ar livre.
%
\paragraph{Time Trial Talks (3T)}
Maratona de mini-palestras dadas por alunos e investigadores da UM, com prémios para as apresentações mais originais.
\begin{description}
\item[Data:] 17.03.2011
\item[Hora:] 16h00
\item[Local:] Anfiteatro do DI
\item[Despesas:] Teremos despesas com os prémios a rondar os 150\euro. Também vamos ter despesas relacionadas com o lanche a oferecer durante o \emph{coffee break}.
\begin{itemize}
\item Prémios
\begin{itemize}
\item 1$\,^{\circ}$ Lugar:
\item 2$\,^{\circ}$ Lugar:
\item 3$\,^{\circ}$ Lugar:
\end{itemize}
\item Apoios
\begin{itemize}
\item Sapo
\item Sun
\item Eurotux
\end{itemize}
\end{itemize}
\end{description}
%
\paragraph{CeSIUMovies}
Iremos realizar uma edição especial do CeSIUMovies realizada em colaboração com o CineFOCUM, que irá ser realizada ao ar livre. Teremos também um prémio para sortear entre os participantes.
\begin{description}
\item[Data:] 17.03.2011
\item[Hora:] 21h30
\item[Local:] Anfiteatro natural (\emph{campus} de Gualtar)
\item[Despesas:] Teremos despesas com a organização e com o prémio para os participantes.
\begin{itemize}
\item Prémios
\begin{itemize}
\item 1$\,^{\circ}$ Lugar:
\item 2$\,^{\circ}$ Lugar:
\item 3$\,^{\circ}$ Lugar:
\end{itemize}
\item Apoios
\begin{itemize}
\item CineFOCUM
\item FNAC
\item \dots
\end{itemize}
\end{itemize}
\end{description}
%sexta feira
\subsubsection*{Sexta-feira}
Último dia de actividades da semana que será dedicado aos antigos estudantes da LESI. O foco principal deste dia é o jantar de antigos alunos. As outras actividades apenas serão realizadas após um estudo de viabilidade a realizar-se num futuro mais próximo do evento.  
%
\paragraph{Mesa Redonda Old-School}
Conversa informal com antigos alunos da LESI sobre a passagem para o mercado do trabalho.
\begin{description}
\item[Data:] 18.03.2011
\item[Hora:] 16h00
\item[Local:] Anfiteatro do DI
\end{description}
%
\paragraph{Jantar Ex-Alunos LESI}
Jantar no restaurante da cantina de Gualtar para todos os antigos alunos da LESI e membros do CeSIUM.
\begin{description}
\item[Data:] 18.03.2011
\item[Hora:] 20h30
\item[Local:] Restaurante Panorâmico 
\end{description}
%
\paragraph{Festa no Insólito}
Festa para os participantes no jantar. Este será o evento que encerra a Semana da LEI.
\begin{description}
\item[Data:] 18.03.2011
\item[Hora:] 24h00
\item[Local:] Insólito Bar
\end{description}
%colocar texto
\subsubsection**{Responsáveis:}
\begin{itemizedash}
\item{Pessoa 1}
\item{Pessoa 2}
\end{itemizedash}
\paragraph{Publico Alvo:}
%texto
O Público Alvo da semana da lei incluí alunos da Licenciatura em Engenharia Informárica, e dos Mestrados em Engenharia Informática, em Informática, e de Sistemas.
\paragraph{Esta actividade já se realizou em anos anteriores?}
A semana da LEI é uma actividade que já é organizada há vários anos. A cada ano que se repete, temos conseguido superar as nossas espectativas, sendo as nossas actividades cada vez mais procuradas pelos alunos.
Para esta edição da Semana da LEI, partimos com uma melhor experiência, resultado das edições anteriores. Partindo do melhor que realizámos no ano anterior, pretendemos este ano conseguir um programa mais completo, e melhor concretização das actividades realizadas ao longo desta semana.
%Sim ou nao
\paragraph{Número de Participantes Esperado:}
%texto
\paragraph{Data:} 14/03/2011 a 18/03/2011
\subsubsection**{Material necessário:}
\begin{itemizedash}
\item{Salas para palestras}
\item{Projectores}
\item{Televisores}
\item{Consolas de jogos}
\end{itemizedash}
\vspace{20pt}
%apoios



\subsubsection*{Parceria AAUM}
A AAUM tem um historial notável no campo do apoio aos núcleos de estudantes. Esse apoio, disponibilizado através do Departamento Social e Núcleos, tem-se traduzido em incentivos à criação de núcleos, suporte logístico e financeiro e a parcerias na realização de actividades mais dinâmicas e pedagógicas, através de um acompanhamento constante e estreita colaboração.

A realização da Semana da LEI II, tal como está planeada, não seria possível sem o apoio da Associação Académica. A AAUM gentilmente dispôs-se a colaborar com o CeSIUM na Semana da LEI, ficando encarregue dos seguintes pontos:

\paragraph{Imagem}
Um dos passos essenciais para o êxito desta iniciativa consiste na criação de uma boa imagem (logotipo, cartazes, \emph{flyers}, etc).

A concepção da imagem da Semana da LEI ficará a cargo da Gen, tirando partido da colaboração já existente com a AAUM. Assim sendo, serão criados vários elementos de divulgação:
\begin{itemize}
\item Logotipo
\item Cartazes
\item \emph{Flyers}
\item T-shirts
\item lonas
\end{itemize}
A par do material com a imagem da Semana da LEI, a AAUM propôs-se ainda disponibilizar capas, canetas e credenciais para os participantes da semana.

\paragraph{Bar Académico}
Alguns dos eventos da Semana realizar-se-ão no Bar Académico de Braga. A AAUM contribuirá autorizando a cedência do espaço, e contactando o concessionário no sentido de fornecerem um barril de cerveja para o Karaoke a realizar-se no primeiro dia da Semana.

\paragraph{Coffee break}
Nos eventos de carácter mais teórico, como a Mesa Redonda Old School e as Time Trial Talks está planeado um intervalo, de forma a evitar que se tornem muito pesadas.
Nestes intervalos será disponibilizado um lanche ligeiro.

Foram consideradas várias hipóteses para a aquisição dos lanches -- pastelaria Cristo Rei (à semelhança do que acontece nas Jornadas de Informática), bares da UM, etc.
Apesar de a AAUM não ter nenhum tipo de parceria para este tipo de serviços, mais uma vez gentilmente se disponibilizaram para tomar a seu cargo esta tarefa.

\paragraph{CeSIUMovies}
Quinzenalmente o CeSIUM tem realizado sessões de cinema. No ano anterior estas decorriam nos anfiteatros do Departamento de Informática; este ano têm decorrido na sede do CeSIUM.

Para a Semana da LEI está prevista uma sessão para a quinta feira à noite, a realizar-se, em princípio, no anfiteatro natural do \emph{campus} de Gualtar, em colaboração com o CineFOCUM.

O CeSIUM dispõe de um projector de vídeo, mas não possui aparelhagem sonora à altura do evento, tendo pedido auxílio à Associação Académica, que se ofereceu para disponibilizar o sistema de som que possui. Contamos também com a ajuda da AAUM para conseguir autorização das entidades responsáveis para realizar esta actividade no anfiteatro natural.

\paragraph{Panorâmico}
O jantar organizado para os ex-alunos de LESI está planeado para decorrer no Restaurante Panorâmico. O CeSIUM conta com o apoio da AAUM na negociação dos preços a serem cobrados pela organização do jantar.





%calendario
\newpage
\subsection{Calendário da Semana}
%\addvspace{2cm}
~\\
\begin{center}
%Aqui fica a tabelinha vunita com o calendario
\begin{tabular}{|c||c|c|c|c|c|c|}
\hline
            & Horas & Segunda  & Terça    & Quarta  & Quinta  & Sexta    \\ \hline\hline
\multirow{6}{*}{Tarde}   & 14h00 &        &        &         &      &        \\ \cline{2-7}
             & 15h00 &       &        &
\multirow{2}{*}{\parbox{2.5cm}{\centering Sessão Pedagógica}}
                                       &
\multirow{5}{*}{\parbox{2.5cm}{\centering Time Trial Talks}}
                                        &
\multirow{5}{*}{\parbox{2.5cm}{\centering Mesa Redonda Old School}}
                                                 \\ \cline{2-4}
             & 16h00 &
\multirow{4}{*}{\parbox{2.5cm}{\centering Torneio de Robocode}}
                       &
\multirow{4}{*}{\parbox{2.5cm}{\centering Torneios de Jogos de Computador}}
                              &      &      &      \\ \cline{2-2} \cline{5-5}
             & 17h00 &      &      &
\multirow{3}{*}{\parbox{2.5cm}{\centering Jogos de Futebol}}
                                       &      &      \\ \cline{2-2}
             & 18h00 &        &        &         &      &        \\ \cline{2-2}
             & 19h00 &        &        &         &      &        \\  \hline\hline


\multirow{6}{*}{Noite}   & 20h00 &       &        &         &      &        \\  \cline{2-7} 
             & 21h00 &
\multirow{4}{*}{\parbox{2.5cm}{\centering Torneios no BA (Matrecos, Bilhar e Ping Pong)}}   
                       &
\multirow{4}{*}{\parbox{2.5cm}{\centering Rally das Tascas}}
                                  &
Jantar de Curso                                &
\multirow{3}{*}{\parbox{2.5cm}{\centering CeSIUMovies}}              &
Jantar ex-LESI                                           \\ \cline{2-2} \cline{5-5} \cline{7-7}
             & 22h00 &        &        &         &      &        \\ \cline{2-2} \cline{5-5} \cline{7-7}
             & 23h00 &        &        &
\multirow{4}{*}{\parbox{2.5cm}{\centering Festão no Sardinha}}
                                  &        &
\multirow{4}{*}{\parbox{2.5cm}{\centering Festa no Insólito}}                 \\ \cline{2-2} \cline{6-6}
             & 00h00 &       &        &         &      &        \\ \cline{2-3} \cline{4-4} \cline{6-6}
             & 01h00 &
\multirow{2}{*}{\parbox{2.5cm}{\centering Karaoke}}   
                       &
\multirow{2}{*}{\parbox{2.5cm}{\centering Festa no BA}}
                                 &         &        &        \\ \cline{2-2} \cline{6-6}
             & 02h00 &        &        &         &      &        \\ \hline
\end{tabular}
\end{center}

%fim semana da lei

