\subsection{Actividade: Seminário ''À lá LESI''} %alterar nome

\subsubsection*{Objectivos da actividade:}
Trazer ao Departamento antigos alunos da LEI, que entraram no mercado de trabalho com a sua empresa, ou entraram numa deste o seu início, para que possam passar o seu testemunho relativamente aos anos que passaram na Universidade, e o porquê de se aventurarem dessa forma no mundo empresarial. Foram escolhidos o Jorge Pereira, o Simão Soares e o Luís Faria, das empresas Seegno, SilicoLife e Keep Solutions, respectivamente.\\
O Jorge, acompanhado pelo Tiago Ribeiro, também ex-LESI, e Rui Marinho, ex aluno de Eng Biomédica, fundou a Seegno, empresa inovadora no sector do webdesign, que tinha sido referida num noticiário da RTP na altura do Seminário, como fonte empreendedora e dinâmica do mercado português. O Simão fundou a SilicoLife, juntamente com professores do ramo Bioinformático e Biológico. Esta foi a empresa vencedora do concurso "Atreve-te 2010", recebendo trinta mil euros de prémio. O Luís não fundou a Keep Solutions, mas esteve nela desde o seu início, e ajudou no desenvolvimento desta Spin-off da UM. Actualmente, a Keep tem enumeros clientes devido aos bons serviços de gestão e organização de documentação.

\subsubsection*{Responsáveis:}
\begin{itemizedash}
	\item{Rui Gonçalo}
	\item{Pedro Faria}
\end{itemizedash}

\paragraph{Publico Alvo: }
Todos os alunos da academia, com particular destaque para os alunos da LEI.

\paragraph{Esta actividade já se realizou em anos anteriores?}
Não

\paragraph{Número de Participantes Esperado:}
50

\paragraph{Data:} 24/11/2010

\subsubsection*{Material necessário:}
\begin{itemizedash}
	\item{Anfiteatro}
	\item{Projector}
\end{itemizedash}

\subsubsection*{Orçamento:}
\begin{center}
\resizebox{0.9\textwidth}{!}{
\begin{tabular}{| c | l | r | l | r |}
\hline
\multirow{2}{*}{Actividade} & \multicolumn{2}{ | c | }{Despesas} & \multicolumn{2}{ | c | }{Receitas} \\ \cline{2-5}
                            & Designação                            & Valor           & Designação                          & Valor             \\ \hline
\multirow{3}{*}{Seminário “à la Lesi” I}
							& Telemóvel e deslocações & 5 \euro		&  								& 					\\ \cline{2-5}
							&						&				&								&					\\ \cline{2-5}
							& \textbf{Total}		& 5 \euro		& \textbf{Total}				& 0 \euro			\\ \cline{1-5} \hline
\end{tabular}
}
\end{center}

\vspace{20pt}
